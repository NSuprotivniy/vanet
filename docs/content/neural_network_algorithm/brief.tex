\subsection*{Алгоритм генеративной нейронной сети}
Генеративно-состязательная сеть (англ. Generative adversarial network, сокращённо GAN)— алгоритм машинного обучения без учителя, построенный на комбинации из двух нейронных сетей, одна из которых (сеть G) генерирует образцы (см. Генеративная модель[en]), а другая (сеть D) старается отличить правильные («подлинные») образцы от неправильных (см. Дискриминативная модель[en]). Так как сети G и D имеют противоположные цели — создать образцы и отбраковать образцы — между ними возникает Антагонистическая игра. Генеративно-состязательную сеть описал Ян Гудфеллоу[en] из компании Google в 2014 году.[1]

Использование этой техники позволяет в частности генерировать фотографии, которые человеческим глазом воспринимаются как натуральные изображения. Например, известна попытка синтезировать фотографии кошек, которые вводят в заблуждение эксперта, считающего её естественными фото.[2] Кроме того GAN может использоваться для улучшения качества нечётких или частично испорченных фотографий.